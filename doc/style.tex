\usepackage[pdftex]{graphicx}
\usepackage{epstopdf}
\usepackage[numbers,sort]{natbib}
\usepackage{setspace}
\usepackage{amsmath, amsthm, amssymb}
\usepackage{scrpage2}
\usepackage{units}
\usepackage{color}
\usepackage{xcolor}
\usepackage{tabularx}

\definecolor{uniblue}{rgb}{0.062745,0.17647,0.34118}
\definecolor{tblue}{HTML}{729FCF}
\definecolor{tgreen}{HTML}{8AE234}
\definecolor{tred}{HTML}{EF2929}

\usepackage{listings}
\lstset{ %
  language=Python,                  % choose the language of the code
  basicstyle=\footnotesize,       % the size of the fonts that are used for the code
  numbers=left,                   % where to put the line-numbers
  numberstyle=\footnotesize,      % the size of the fonts that are used for the line-numbers
  stepnumber=1,                   % the step between two line-numbers. If it's 1 each line
                                  % will be numbered
  numbersep=5pt,                  % how far the line-numbers are from the code
  backgroundcolor=\color{white},  % choose the background color. You must add \usepackage{color}
  showspaces=false,               % show spaces adding particular underscores
  showstringspaces=false,         % underline spaces within strings
  showtabs=false,                 % show tabs within strings adding particular underscores
  %frame=single,                  % adds a frame around the code
  tabsize=4,                      % sets default tabsize to 2 spaces
  captionpos=b,                   % sets the caption-position to bottom
  breaklines=true,                % sets automatic line breaking
  breakatwhitespace=false,        % sets if automatic breaks should only happen at whitespace
  %title=\lstname,                % show the filename of files included with \lstinputlisting;
                                  % also try caption instead of title
  escapeinside={\%*}{*)},         % if you want to add a comment within your code
  morekeywords={*,...}            % if you want to add more keywords to the set
}

\setcounter{tocdepth}{3}
\renewcommand{\arraystretch}{1.5}

\usepackage[absolute]{textpos}

\newlength{\TitleMargin}
\newlength{\TitleWidth}

\setlength{\TitleMargin}{2cm}
\setlength{\TitleWidth}{\paperwidth}
\addtolength{\TitleWidth}{-\TitleMargin}
\addtolength{\TitleWidth}{-\TitleMargin}


\newcommand{\TitleUni}{University of Potsdam}
\newcommand{\TitleInstitut}{Faculty of Science}
\newcommand{\TitleTitel}{IBM Tivoli Workload Scheduler LoadLeveler module for energy-saving daemon Cherub}
\newcommand{\TitleTyp}{Internship report}
\newcommand{\TitleAutor}{Sebastian Menski\\\texttt{menski@uni-potsdam.de}}
\newcommand{\TitleBetreuerText}{Supervisor}
\newcommand{\TitleBetreuer}{Simon Kiertscher \\ &Ciaron Linstead}
\newcommand{\TitleAbschlussText}{~}
\newcommand{\TitleOrt}{Potsdam}
\newcommand{\TitleDatum}{June 27, 2013}

\renewcommand{\maketitle}{
  \thispagestyle{empty}
  \begin{textblock*}{\TitleWidth}(\TitleMargin,\TitleMargin)
    ~\hfill\includegraphics[height=2.5cm]{images/uni-logo}\\[3mm]
    {\color{uniblue}\rule{\TitleWidth}{1mm}}\\[5mm]
    {
      \centering
      \sffamily\Large
      {\LARGE\TitleUni}\\[0.5\baselineskip]
      {\large\TitleInstitut}\\[5\baselineskip]
      {\Huge\TitleTitel}\\[3\baselineskip]

      {\TitleTyp}\\
      [2\baselineskip]

      \TitleAutor\\[3\baselineskip]
      \begin{tabular}{rl}
        \TitleBetreuerText: & \TitleBetreuer
      \end{tabular}\\[2\baselineskip]
      \TitleOrt, \TitleDatum\par
    }
  \end{textblock*}~\clearpage{}
}
